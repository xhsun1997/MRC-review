\section{Introduction}
Natural language processing(NLP) is an important direction in the field of computer science and 
artificial intelligence, which studies various theories and technologies that can realize effective 
communication between human and computer with natural language. Machine reading comprehension(MRC) is to let 
machine learn to read and understand natural language text, find the answer from the relevant articles of a given question.
MRC is one of the most challenging tasks in the field of NLP, which has many application scenarios, 
such as intelligent Q\&A in search engine, intelligent customer service in e-commerce field and so on.

As early as the 1970s, scholars have realized that machine reading technology is the key method to test 
computer understanding of human language, such as the QUALM system built by Lehnert. Due to the manual coding and  the system is very small, 
it is difficult to be extended to a larger field. 
% 基于深度学习的机器阅读理解综述,李舟军 文献4-6: Deep read: a reading comprehension system.(HIRSCHMAN等人)
% A rule-based question answering system for reading comprehension tests(RILOFF等人)
% Machine reading at the university of washington(POON等人)
%Mctest: A challenge dataset for the open-domain machine comprehension of text(RICHARDSON等人)
Hirschman built the first automatic reading comprehension test system deep read in 1999. The system measures 
reading comprehension tasks based on stories, and uses bad-of-word model and manual rules for pattern matching, 
accuracy rate can reach about 40\%. Due to the manual rules, however, the generalization ability of the model is poor.
Riloff scores the matching degree of questions and candidate sentences in the article by making rules manually, then 
select the candidate sentence with the highest score as the answer.
The traditional MRC technology mostly uses pattern matching method to extract features, even if using machine learning method, 
it only answers question at the sentence level granularity, and can only extract shallow features from the text. 
Therefore, the early MRC system has poor performance and it is difficult to put it into practical application, which 
leads to the slow development of MRC field.

With the rise of deep learning and development of technology in NLP field, in order to make up for the defects of 
the traditional MRC technology, Hermann, a researcher at deepmind, used neural network model to solve MRC tasks in 2015 and 
constructed a reading comprehension dataset CNN\&Daily Mail which is larger than the previous datasets.
They proposed two models : Attentive Reader and Impatient Reader, which based on the neural network and attention mechanism, 
and the performance of these two models on CNN\&Daily Mail is much better than traditional models.
This work is regarded as the foundation work in the field of MRC. Since then, more and more scholars 
have constructed better models based on these two models. 
For example, Kadlec use dot product to simplify attention operation and achieves better performance. Chen uses 
bilinear function as activation function to obtain more flexible and effective model. Cui enhances the ability of feature extraction by using 
attention mechanism between article and question, which attention mechanism is used not only in article, but also in question.
Because the models are based on neural network, they are also refered to as neural machine reading comprehension models.

This paper reviews the research tasks, related datasets and models in the field of MRC since 2015.